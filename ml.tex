
\documentclass{article}
\usepackage{amsmath}
\usepackage{amsfonts}

\author{a1trl9}
\title{Machine Learning Review}
\date{}

\setlength\parindent{0pt}
\counterwithin*{equation}{section}

\newcommand{\V}[1]{\boldsymbol{#1}}
\newcommand{\TM}{\bigtriangledown}

\begin{document}
\maketitle
\section{Regression}
\subsection{Linear Regression}
\subsection{Logistic Regression}
\subsubsection{Why Logistic Regression is a Linear Classifier}

Noting:
\begin{equation}
P(y_i=1)=\frac{1}{exp(-\V{w'}\V{x_i})+1}
\end{equation}

If we set \(p=0.5\) as the threshold, then:
\begin{equation}
\left\{
\begin{aligned}
&\hat{y}=1 \quad if \quad \V{w'}\V{x}>0\\
&\hat{y}=0 \quad if \quad \V{w'}\V{x}\leq0
\end{aligned}
\right.
\end{equation}

It yields that if all training data are linearly separable, logistic regression
may lead to over-fitting. Intuitively, when \(\V{w}\) subjects to all data are
classified correctly, scaling \(\V{w}\) up can always increase likelihood and when
\(\V{w} \to \infty\), \(p(y_i)=1\) for all data points.

\subsubsection{The convexity of Negative Log-likelihood Function}

Since:
\begin{equation}
\left\{
\begin{aligned}
&P(y_i=1)=\frac{1}{exp(-\V{w'}\V{x_i})+1}\\
&P(y_i=0)=\frac{1}{exp(\V{w'}\V{x_i})+1}
\end{aligned}
\right.
\end{equation}
Where \(\V{w}\) is the coefficient vector while \(\V{x_i}\) is the feature vector.
Assuming the dimension is \(m\).

If we transform the label set from \(\{1, 0\}\) to \(\{1, -1\}\), then
\begin{equation}
p(y_i)=\frac{1}{exp(-y_i\V{w'}\V{x_i})+1}
\end{equation}

Let \(z=-y_i\V{w'}\V{x_i}\), then the negative log-likelihood function is:
\begin{equation}
-L=-\sum_{i}^{N}log[\frac{1}{exp(z)+1}]
\end{equation}

Here we assume \(L\) is \textbf{twice continuously differentiable}.

Let \(g(z)=\frac{1}{exp(z)+1}\), then:
\begin{equation}
g'(z)=-\frac{exp(z)}{[exp(z)+1]^2}=-g(z)[1-g(z)]
\end{equation}

Therefore:
\begin{equation}
\begin{aligned}
\frac{d(-L)}{dw_j}&=-\sum_{i}^N\frac{1}{g(z)}g(z)[1-g(z)]y_ix_{ij}\\
&=-\sum_{i}^Ny_ix_{ij}[1-g(z)]
\end{aligned}
\end{equation}

And:
\begin{equation}
\begin{aligned}
\frac{d(-L)}{dw_jdw_k}&=\sum_{i}^Ny_i^2x_{ij}x_{ik}g(z)[1-g(z)]\\
&=\sum_{i}^Nx_{ij}x_{ik}g(z)[1-g(z)]
\end{aligned}
\end{equation}

Let \(\V{a}\) be any m-dimension vector. And \(\TM^2\) is the Hessian matrix
of \(-L\). Then:

\begin{equation}
\begin{aligned}
\V{a'}\TM^2\V{a}&=\sum_{i}^N\sum_{j}^m\sum_{k}^ma_ja_kx_{ij}x_{ik}g(z)[1-g(z)]\\
&=\sum_{i}^N{\{\sqrt{g(z)[1-g(z)]}\V{a'}\V{x_i}\}^2}\\
&\geq 0
\end{aligned}
\end{equation}

So \(\TM^2\) is a positive semi-definite matrix, and \(-L\) is Therefore
a convex function.

\vspace{2mm}
\textbf{Further}
\begin{itemize}
\item It can be proved, that for any identifiable model from exponential family,
the negative log-likelihood function is strictly convex and therefore
the maximum likelihood estimate is unique \textbf{if it exists}. When model
is not identifiable (happens when variables are not independent), only
convex holds.
\item When \(x_1, x_2, \cdots, x_N\) are not linearly independent, it is possible
that \(\V{a'}\TM\V{a}=0\), in which case, \(-L\) is not strictly convex and
the minimum might not be unique.
\item If we add L2 regularizer, then \(-L\) will be strictly convex.
\end{itemize}

\subsubsection{*Convexity \& Concavity and Hessian Matrix}

Suppose \(f: V \rightarrow \mathbb{R} \) is a twice differentiable function. \(V\) is
a convex set on \(\mathbb{R}^n\). If \(f\) is not convex, then \(\exists (\V{x_1}, 
\V{x_2}) \in V, \exists t \in (0, 1) \), \(f((1-t)\V{x_1} + t\V{x_2}) > (1-t)f(\V{x_1})+tf(\V{x_2})\).
Let \(\theta: (0, 1) \rightarrow R: \theta(k)=f((1-k)\V{x_1}+k\V{x_2})\). Then
\begin{equation}
\theta(t)=f((1-t)\V{x_1}+t\V{x_2}) > (1-t)f(\V{x_1})+tf(\V{x_2})=(1-t)\theta(0)+t\theta(1)\
\end{equation}

Therefore:
\begin{equation}
\left\{
\begin{aligned}
&\theta(t)-\theta(0)>t(\theta(1)-\theta(0))\\
&\theta(1)-\theta(t)<(1-t)(\theta(1)-\theta(0))
\end{aligned}
\right.
\end{equation}

According to Mean Value Theorem, \(\exists t_1 \in (0, t), t_2 \in (t, 1)\),
subject to \(\frac{\theta}{dt}(t_1)t=\theta(t)-\theta(0),
\frac{\theta}{dt}(t_2)(1-t)=\theta(1)-\theta(t)\). So:

\begin{equation}
\left\{
\begin{aligned}
&\frac{\theta}{dt}(t_1)>\theta(1)-\theta(0)\\
&\frac{\theta}{dt}(t_2)<\theta(1)-\theta(0)
\end{aligned}
\right.
\end{equation}

It further yields:

\begin{equation}
\frac{\theta}{dt}(t_1)>\frac{\theta}{dt}(t_2)
\end{equation}

Apply Mean Value Theorem again:
\begin{equation}
\frac{\theta}{dt}(t_2)-\frac{\theta}{dt}(t_1)=(t_2-t_1)\frac{d^2\theta}{dt^2}(t_3)
\end{equation}

Where \(t_3 \in (t_1, t_2)\). And obviously \(\frac{d^2\theta}{dt^2}(t_3)<0\).

As we know \(\frac{d^2\theta}{dt^2}=(\V{x_2}-\V{x_1})'d^2f(\V{x_2}-\V{x_1})\),
where \(d^f\) is the Hessian matrix of \(f\). When it is positive semi-definite,
\(\V{x_2}-\V{x_1})'d^2f(\V{x_2}-\V{x_1})\geq 0\), which is contradict to
\(\frac{d^2\theta}{dt^2}(t_3)<0\).

Therefore, when the Hessian matrix of \(f\) is positive semi-definite, \(f\) is
convex. Similarly, if the Hessian matrix of \(f\) is negative semi-definite,
\(f\) is concave.

\section{Regularisation}
\subsection{Transformation for Linear Regression}
Without regularisation:
\begin{equation}
L=||\mathbf{X}\V{w}-\V{y}||^2
\end{equation}

With regularisation:
\begin{equation}
L=||\mathbf{X}\V{w}-\V{y}||^2+\lambda||\V{w}||^2
\end{equation}

Where \(\lambda\) controls the strength of regularisation. Now we try to solve:
\begin{equation}
\begin{aligned}
&2\mathbf{X}'(\mathbf{X}\V{w}-\V{y})+2\lambda \V{w}=\V{0}\\
&\implies (\mathbf{X}'\mathbf{X}+\lambda \mathbf{I})\V{w}=\mathbf{X}'\V{y}\\
&\implies \V{w}=(\mathbf{X}'\mathbf{X}+\lambda \mathbf{I})'\mathbf{X}'\V{y}
\end{aligned}
\end{equation}

As we know \(\mathbf{A}=\mathbf{X}'\mathbf{X}\) is a symmetric matrix and for any non-zero vector \(V{v}\)
(same dimension as \(\mathbf{A}\)):
\begin{equation}
\V{v'}\mathbf{A}\V{v}=\V{v'}\mathbf{X}'\mathbf{X}\V{v}=(\mathbf{X}\V{v})'\mathbf{X}\V{v}\geq 0
\end{equation}

Then \(\mathbf{A}\) is a positive semi-definite matrix. Besides, consider
any eigenvector / eigenvalue pair \(\V{p}, t\) for \(\mathbf{A}\):
\begin{equation}
\begin{split}
(\mathbf{A}+\lambda\mathbf{I})\V{p}=\mathbf{A}\V{p}+\lambda\V{P}
=t\V{p}+\lambda\V{p}=(t+\lambda)\V{p}
\end{split}
\end{equation}

Therefore, for all eigenvalues \(t_1, t_2, \cdots, t_n\) for \(\mathbf{A}\),
\(t_1+\lambda, t_2+\lambda, \cdots, t_n+\lambda\) is eigenvalues for
\(\mathbf{A}+\lambda\mathbf{I}\). Since \(t_1, t_2, \cdots, t_n\) have all been
non-negative as \(\mathbf{A}\) is positive semi-definite, and \(\lambda>0\), \(t_1+\lambda, t_2+\lambda, \cdots, t_n+\lambda\)
are all positive. As the multiple of \(n\) eigenvalues is equal to the
determinant of \(\mathbf{A}\), \(det(\mathbf{A})\neq 0\) and \(rank(\mathbf{A})=n\),
\(\mathbf{X}'\mathbf{X}+\lambda\mathbf{I}\) must be invertable.

\subsection{Transformation for Logistic Regression}

Without regularisation (ignoring intercept):
\begin{equation}
\begin{aligned}
&L=-\sum_{i}^my_ilog(\sigma(\V{x}_i\V{w}))+(1-y_i)log(1-\sigma(\V{x}_i\V{w}))\\
&\implies
\frac{dL}{\V{w}}=-\sum_{i}^m(y_i(1-\sigma(\V{x}_i))-(1-y_i)\sigma(\V{x}_i\V{w})))\V{x}'_i\\
&=\sum_{i}^m(\sigma(\V{x}_i\V{w})-y_i)\V{x}_i'\\
\end{aligned}
\end{equation}

With regularisation:
\begin{equation}
\begin{aligned}
&\frac{dL}{\V{w}}=-\sum_{i}^m(y_i(1-\sigma(\V{x}_i))-(1-y_i)\sigma(\V{x}_i\V{w})))\V{x}'_i+\lambda\V{w}\\
&=\sum_{i}^m(\sigma(\V{x}_i\V{w})-y_i)\V{x}_i'+\lambda\V{w}\\
\end{aligned}
\end{equation}

\section{Perceptron}
\subsection{Perceptron Convergence}
Assuming the training data (\(N\) points) are linearly separable, in which case, 
\(\exists \V{v}: ||\V{v}||=1\), subjects to \(y_i\V{v'}\V{x_i}>\gamma\), for
\(i=1, 2, \cdots N\).
Noting at each gradient step, the update is in the form of:

\begin{equation}
\V{w}^{t+1}=\V{w}^t+\eta y_j\V{x_j}
\end{equation}

Where \(<y_j, \V{x_j}>\) is the pair of predicted class \& selected misclassified point.

As we know:
\begin{equation}
||y_k\eta\V{x_k} +\V{w}^{k-1}||^2=||y_k\eta\V{x_k}||^2+||\V{w}^{k-1}||^2+y_k\eta\V{x_k}\cdot \V{w}^{k-1}
\end{equation}

Moreover, \(y_k\V{x}\V{w}^{k-1}\) is always negative since \(\V{x_k}\) is a
misclassified point. Therefore:

\begin{equation}
\begin{aligned}
||\V{w}^k||^2&=||y_k\eta\V{x_k} +\V{w}^{k-1}||^2<||y_k\eta \V{x_k}||^2+||\V{w}^{k-1}||^2\\
&\leq (\eta R)^2 + ||\V{w}^{k-1}||^2
\end{aligned}
\end{equation}

Iteratively, we get:

\begin{equation}
||\V{w}^k||^2<||\V{w}^0||^2+k(\eta R)^2
\end{equation}

Meanwhile:
\begin{equation}
\begin{aligned}
\sum_{i=1}^k|\V{v'}\cdot (\V{w}^i-\V{w}^{i-1})|&=\V{v'}\cdot\sum_{i=1}^k(\V{w}^i-\V{w}^{i-1})\\
&=\V{v'}\cdot (\V{w}^k-\V{w}^0)\\
&\leq ||\V{v'}||||\V{w}^k-\V{w}^0||\\
&=||\V{w}^k-\V{w}^0||\\
&= \sqrt{||\V{w}^k||^2 + ||\V{w}^0||^2 - \V{w}^k \cdot \V{w}^0}\\
&\leq \sqrt{||\V{w}^k||^2+||\V{w}^0||^2+||\V{w}^k||||\V{w}^0||}
\end{aligned}
\end{equation}

In addition, since:
\begin{equation}
\begin{aligned}
\sum_{i=1}^k|\V{v'}\cdot (\V{w}^i-\V{w}^{i-1})|&=\sum_{i=1}^k|\V{v'}\cdot (y_i\eta\V{x_i}')|\\
&\geq ky
\end{aligned}
\end{equation}

We finally get:
\begin{equation}
\begin{aligned}
k\gamma &\leq \sum_{i=1}^k|\V{v'}\cdot (\V{w}^i-\V{w}^{i-1})|\\
&\leq \sqrt{||\V{w}^k||^2+||\V{w}^0||^2+||\V{w}^k||||\V{w}^0||}\\
&\leq\sqrt{||\V{w}^0||^2+k(\eta R)^2+||\V{w}^0||^2+||\V{w}^0||\sqrt{||\V{w}^0||^2+k(\eta R)^2}}
\end{aligned}
\end{equation}

Here, we treat \(\V{w}^0\) as constant and therefore:

\begin{equation}
k\gamma < \sqrt{c_1}\sqrt{k(\eta R)^2}
\end{equation}

So \(k<O((\frac{\eta R}{\gamma})^2)\)

\section{Neural Network}
\subsection{Local Quadratic Approximation}
Consider the Taylor expansion of the error function \(E(\V{w})\) around
a point \(\hat{\V{w}}\) in the weight space:
\begin{equation}
E(\V{w})=E(\hat{\V{w}})+(\V{w}-\hat{\V{w}})'\V{b}
+\frac{1}{2}\hat{\V{w}}'\V{H}\hat{\V{w}}
\end{equation}

Where \(\V{H}\) is the Hessian matrix at \(\hat{\V{w}}\).

First, let \(V\) be the n-dimension linear space for the linear
transformation \(H\). As \(\V{H}\) is symmetric, we can construct \(n\) orthonormal eigenvectors.

\vspace{2mm}
\textbf{Proof}

Let \(\lambda_1, \lambda_2\) be two eigenvalues of \(\V{H}\), so we have:
\begin{equation}
\begin{split}
&\V{H}\V{p_1}=\lambda_1\V{p_1}\\
&\V{H}\V{p_2}=\lambda_1\V{p_2}\\
\end{split}
\end{equation}

So:
\begin{equation}
\begin{split}
(\lambda_1\V{p_1})'\V{p_2}=(\V{H}\V{p_1})'\V{p_2}
=\V{p_1}'\V{H}'\V{p_2}
\end{split}
\end{equation}

As \(\V{H}\) is symmetric, \(\V{H}'=\V{H}\):
\begin{equation}
\lambda_1\V{p_1}\V{p_2}=
\V{p_1}'\V{H}'\V{p_2}=\V{p_1}'\V{H}\V{p_2}=\V{p_1}'\lambda_2\V{p_2}
=\lambda_2\V{p_1}\V{p_2}
\end{equation}

Now consider two cases:

First,  
\(\lambda_1, \lambda_2\) are distinct eigenvalues, in this case
\(\V{p_1}, \V{p_2}\) are orthogonal.

\vspace{2mm}
Second, if \(\lambda_1, \lambda_2\) are the same eigenvectors. Then
build a pair of orthogonal eigenvectors \(\alpha_1\V{p_1} + \beta_1\V{p_1}\)
and \(\alpha_2\V{p_1}+\beta_2\V{p_2}\) in the eigenvector space
\(\{\V{a}|\V{H}\V{a}=\lambda_1\V{a}\}\). Since \(\V{H}\) is diagonalizable,
direct sum of eigenvector spaces equals to \(V\), the linear representation of \(\alpha_1\V{p_1} + \beta_1\V{p_1}\)
and \(\alpha_2\V{p_1}+\beta_2\V{p_2}\) are unique and therefore they are 
linearly independent
with eigenvectors paired with eigenvalues \(\lambda\neq \lambda_1\). Combining
all these eigenvectors could finally construct an \(n\times n\) orthogonal eigenvector matrix 

\vspace{2mm}
For both cases, we could then normalize the eigenvector matrix.

We can use these \(n\) orthonormal eigenvectors as a basis in the
linear space \(V\). And present \(\hat{\V{w}}\) as
\(\sum_{i=1}^n\alpha_i\V{p}_i\) where \(\V{p}_i\) is the eigenvector.
Now:
\begin{equation}
\begin{split}
\hat{\V{w}}'\V{H}\hat{\V{w}}
&=(\sum_{i=1}^n\alpha_i\V{p}_i)'\V{H}(\sum_{i=1}^n\alpha_i\V{p}_i)\\
&=(\sum_{i=1}^n\alpha_i\V{p}_i)'\sum_{i=1}^n
\alpha_i\V{H}\V{p_i}\\
&=(\sum_{i=1}^n\alpha_i\V{p}_i)'\sum_{i=1}^n\alpha_i\lambda_i\V{p_i}\\
&=\sum_{i=1}^n\sum_{j=1}^n\alpha_i\alpha_j\lambda_j\V{p}_i'\V{p}_j\\
&=\sum_{i=1}^n\alpha_i^2\lambda_i\V||{p}_i||^2\\
&=\sum_{i=1}^n\lambda_i\alpha_i^2
\end{split}
\end{equation}

\subsection{Numerical Solution -- Finite Difference}
\subsubsection{Why central difference is more precise?}
For one-side difference, with Taylor expansion:
\begin{equation}
f(x+\epsilon)=f(x)+\epsilon f'(x)+o(\epsilon)
\end{equation}
So:
\begin{equation}
\frac{f(x+\epsilon)-f(x)}{\epsilon}=f'(x)+o(\epsilon)
\end{equation}

For central difference, with Taylor expansion:
\begin{equation}
\begin{split}
&f(x+\epsilon)=f(x)+\epsilon f'(x)+\epsilon^2\frac{f''(x)}{2!}
+o(\epsilon^2)\\
&f(x-\epsilon)=f(x)-\epsilon f'(x)+\epsilon^2\frac{f''(x)}{2!}
+o(\epsilon^2)
\end{split}
\end{equation}
Therefore:
\begin{equation}
\begin{split}
&f(x+\epsilon)-f(x-\epsilon)=2\epsilon f'(x)+o(\epsilon^2)\\
&\implies \frac{f(x+\epsilon)-f(x-\epsilon)}{2\epsilon}=f'(x)+o(\epsilon^2)
\end{split}
\end{equation}

\end{document}
