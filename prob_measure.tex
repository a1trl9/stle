\documentclass{article}

\usepackage[a4paper, total={6in, 10in}]{geometry}
\usepackage{graphicx}
\usepackage{enumitem}
\usepackage{hyperref}
\usepackage{amsmath}
\usepackage{amsfonts}

\newcommand{\mc}{\mathcal}
\newcommand{\mb}{\mathbb}
\DeclareMathOperator{\sialge}{\sigma-algebra}
\DeclareMathOperator{\sifin}{\sigma-finite}
\DeclareMathOperator{\pisys}{\pi-system}
\DeclareMathOperator{\picls}{\pi-class}
\DeclareMathOperator{\lamsys}{\lambda-system}
\DeclareMathOperator{\lamcls}{\lambda-class}

\title{Notes on Measure Theory \& Probability}
\author{a1trl9}
\date{}
\setlength{\parindent}{0cm}

\begin{document}

\maketitle

\section{Measure}
\subsection{Classes of sets}
\textbf{Definition 1.1.1:} let \(\Omega\) be a nonempty set and \(\mc{P}(\Omega)\) be the power set of \(\Omega\).
A collection of sets \(\mc{F}\subset \mc{P}(\Omega)\) is called an \textit{algebra} if

\begin{enumerate}[label=(\alph*)]
\item \(\Omega \in \mc{F}\).
\item \(A \in \mc{F}\) implies \(A^c \in \mc{F}\).
\item \(A, B \in \mc{F}\) implies \(A \cup B \in \mc{F}\).
\end{enumerate}

It is easy to see with (b) (\textit{symmetric}), (c) is equivalent to

\[(c)' \quad A, B \in \mc{F} \implies A \cap B \in \mc{F} \]

\textbf{Proof:} to prove (c) \(\implies\) (c)', let \(A, B \in \mc{F}\), by (b), \(A^c, B^c \in \mc{F}\). By (c), \(
A^c \cup B^c \in \mc{F}\). By (b) again, \(\Omega \setminus (A^c \cup B^c) = (A^c \cup B^c)
\in \mc{F}\). \(\Omega \setminus (A^c \cup B^c) = (\Omega \setminus A^c) \cap (\Omega
\setminus B^c) = A \cap B\). Proved. The converse proof follows similar steps.

\vspace{2mm}
\textbf{Definition 1.1.2: } a class \(\mc{F} \subset \mc{P}(\Omega)\) is called an \textit{\(\sialge\)} if 
it is an \textit{algebra} and

\[A_n \in \mc{F} \quad for \quad n \geq 1 \implies \bigcup_{n\geq 1}A_n \in \mc{F}\]

\textbf{Proposition 1.1.1: } a class \(\mc{F} \subset \mc{P}(\Omega)\) is a \(\sialge\) \textit{iff} 
\(\mc{F}\) is 
an \textit{algebra} and satisfies

\[A_n \in \mc{F}, A_n \subset A_{n+1} \quad for \: all \quad n \implies \bigcup_{n \geq 1}A_n \in \mc{F} \]

\textbf{Proof:} to prove \textit{if}, let \(\{B_n\}_{n\geq 1} \in \mc{F}\), then for all \(n\geq 1\),
\(\bigcup_{i=1}^nB_i \in \mc{F}\) (by (c)) and obviously \(\bigcup_{i}^{n-1}B_n \subset \bigcup_{i}^{n}B_n\).
Let \(A_n=\bigcup_{i=1}^nB_i\), easy to see \(\bigcup_{n\geq 1}A_n=\bigcup_{n\geq 1}B_n\). With the condition given, 
\(\bigcup_{n\geq 1}B_n \in \mc{F}\). Proved. The \textit{only if} part is obvious.

\vspace{2mm}
Examples of \textit{algebra} or \(\sialge\) (or neither):

\begin{enumerate}
\item \(\mc{F}_1=\mc{P}(\Omega)\equiv\{A: A \subset \Omega\}\) (\(\sialge\)).
\item \(\mc{F}_2= \{\emptyset, \Omega \} \) (\(\sialge\), also called \textit{trival \(\sialge\)}).
\item \(\mc{F}_3= \{A \subset \Omega\: :\: either\: |A|\: is\: finite\: or\: |A^c|\: is\: finite \} \) (\textit{algebra} but 
\item \(\mc{F}_4= \{A \subset \Omega\: :\: either\: |A|\: is\: countable\: or\: |A^c|\: is\: countable \} \) ( 
\(\sialge\)).
\end{enumerate}

For (3), if \(\Omega\) is infinite, let \({A_1, A_2, \cdots, A_n} \subset \Omega\) so that for all \(A_i, i\geq 1, A_i \in \mc{F}_3\)
. However both \(A\equiv \bigcup_{i\geq 1} A_{2i-1}\) and \(A^c\) are infinite, so \(A \not\in \mc{F}_3\).

\vspace{2mm}
For (4), suppose \(A_1, A_2, \cdots, A_n \subset \Omega\). If at least one of \(A_i, i\geq 1\) is countable, then \(\bigcap_{i\geq 1}A_n\)
is countable and \(\bigcap_{i\geq 1}A_n \in \mc{F}_4\). If all \(A_i \geq 1\) are uncountable, then all \(A_i^c \geq 1\) are
countable. \((\bigcup_{i\geq 1}A_n)^c=(\bigcap_{i\geq 1}A_i^c) \in \mc{F}_4\). As (b) in definition 1.1.1 is easy to prove, \(\bigcup_{i\geq 1}A_n \in \mc{F}_4\),
which is equivalent to \(\bigcap_{i\geq 1}A_n\in \mc{F}_4\). 

\vspace{2mm}
\textbf{Definition 1.1.3: } if \(\mc{A}\) is a class of subsets of \(\Omega\), then the \(\sialge\) generated by \(\mc{A}\)
, denoted by \(\sigma(\mc{A})\) is defined as:

\[
\sigma(\mc{A}) = \bigcap_{\mc{F}\in \mc{I}(\mc{A})}\mc{F}   
\]

where \(\mc{A}\equiv \{\mc{F}: \mc{A} \subset \mc{F}\) and \(\mc{F}\) is a \(\sialge\) on  \(\Omega\}\).

\vspace{2mm}
\textbf{Definition 1.1.4: } The \textit{Borel \(\sialge\)} on a topological space \(\mb{S}\) is defined as the \(\sialge\) generated
by the collection of open sets in \(\mb{S}\).

\vspace{2mm}
\textbf{Definition 1.1.5: } A class \(\mc{C}\) of subsets of \(\Omega\) is a \(\pisys\) or a \(\picls\) if
\(A, B \in \mc{C} \implies A \cap B \in \mc{C}\).

\vspace{2mm}
\textbf{Definition 1.1.6: } A class \(\mc{L}\) of subsets of \(\Omega\) is a \(\lamsys\) or a \(\lamcls\) if:

\begin{enumerate}[label=(\roman*)]
\item \(\Omega \in \mc{L}\).
\item \(A, B \in \mc{L}, A \subset B, \implies B\setminus A \in \mc{L} \).
\item \(A_n \in \mc{L}, A_{n} \subset A_{n+1}\) for all \(n \geq 1 \implies \bigcup_{n\geq 1}A_n \in \mc{L}\).
\end{enumerate}

It is easy to that every \(\sialge\) is a \(\lamsys\). The converse is not necessarily true.

\vspace{2mm}
\textbf{Theorem 1.1.2: }if \(\mc{C}\) is a \(\pisys\), then \(\lambda(\mc{C})=\sigma(\mc{C})\).

\vspace{2mm}
\textbf{Proof: } As mentioned, every \(\sialge\) of \(\mc{C}\) is a \(\lamsys\). Therefore, \(\lambda(\mc{C}) \subset 
\sigma(\mc{C})\). Suppose the class of subsets of \(\mc{C}\): \(\lambda_1 \equiv \{A: A\in \lambda(\mc{C})\; |\; A \cap B \in \lambda(\mc{C})
\) for all \(B\in \lambda(\mc{C})\}\). It is obvious that \(\Omega \in \lambda_1(\mc{C})\). If \(A_1, A_2\in \lambda_1(\mc{C})\),
\(A_1\subset A_2\), then for all \(B\in \lambda(\mc{C})\), \((A_2\setminus A_1)\cap B = (A_2\setminus B)\cap (A_1\setminus B)\).
As both \(A_2\setminus B, A_1\setminus B \in \lambda(\mc{C})\), \((A_2\setminus B)\cap (A_1\setminus B)\in \lambda(\mc{C})\),
indicating \(A_2\setminus A_1\in \lambda_1(\mc{C})\). If \(A_1, A_2, \cdots, A_n \in \lambda_1(\mc{C})\),
\(A_n\subset A_{n+1}\) for \(n\geq 1\), then for all \(B\in \lambda(\mc{C})\), \((\bigcup_{n\geq 1}A_n)\cap B=\bigcup_{n\geq 1}(A_n\cap B)\),
as \(A_n\cap B \in \lambda(\mc{C})\), \((A_n\cap B)\subset (A_{n+1}\cap B)\), \(\bigcup_{n\geq 1}(A_n\cap B)\in \lambda(\mc{C})\), indicating
\(\bigcup_{n\geq 1}A_n \in \lambda_1(\mc{C})\). Therefore, \(\lambda_1(\mc{C})\) is a \(\lamsys\).

Meanwhile, suppose \(\lambda_2 \equiv \{A: A\in \lambda(\mc{C})\: |\: A \cap B \in \lambda(\mc{C})\) for all \(B\in \mc{C}\}\).
Obviously \(\lambda_1(\mc{C})\subset \lambda_2(\mc{C})\).
As \(\mc{C}\subset \lambda(\mc{C})\), it is easy to prove \(\lambda_2\) is also a \(\lamsys\). Besides,
as \(\mc{C}\) is a \(\pisys\), if \(A_1, A_2 \in \mc{C}\), \(A_1\cap A_2 \in \mc{C}\subset \lambda(\mc{C})\), indicating
\(\mc{C}\subset \lambda_2(\mc{C})\). While \(\lambda_2(\mc{C})\subset \lambda(\mc{C})\), \(\lambda_2(\mc{C})=\lambda(\mc{C})\).
Therefore, \(A\cap B\in \lambda(\mc{C})\) for each pair \(A\in \mc{C}, B\in \lambda(\mc{C})\), i.e. \(\mc{C}\subset \lambda_1(\mc{C})\).
While \(\lambda_1(\mc{C})\subset \lambda(\mc{C})\), \(\lambda_1(\mc{C})=\lambda(\mc{C})\). So \(\lambda(C)\) is close under intersection.
As being a \(\lamsys\) guanratees the other two conditions, \(\lambda(\mc{C})\) is also a \(\sialge\). With \(\lambda(\mc{C}\subset
\sigma(\mc{C}))\), proved.

\subsection{Measures}

\textbf{Definition 1.2.1: } Let \(\Omega\) be a nonempty set and \(\mc{F}\) be an algebra on \(\Omega\). Then a function
\(\mu\) on \(\mc{F}\) is called a measure if:

\begin{enumerate}[label=(\roman*)]
\item \(\mu(A)\in [0, \infty]\) for all \(A \in \mc{F}\).
\item \(\mu(\emptyset)=0\).
\item for any disjoint collection of sets \(A_1, A_2, \cdots, A_n \in \mc{F}\) with \(\bigcup_{n\geq 1}A_n\in \mc{F}\),
\(\mu(\bigcup_{n\geq 1}A_n)=\sum_{n=1}^{\infty}\mu(A_n)\).
\end{enumerate}

\vspace{2mm}
\textbf{Proposition 1.2.1: } Let \(\Omega\) be a nonempty set and \(\mc{F}\) be an algebra of subsets of \(\Omega\)
and \(\mu\) be a set function on \(\mc{F}\) with values in \([0, \infty]\) and with \(\mu(\emptyset)=0\).
Then \(\mu\) is a measure \textit{iff} \(\mu\) satisfies:

\begin{enumerate}[label=(\roman*)\('\)]
\item (finite additivity) for all \(A_1, A_2 \in \mc{F}\) with \(A_1\cap A_2 = \emptyset\), 
\(\mu(A_1\cup A_2)=\mu(A_1)+\mu(A_2)\).
\item (monotone continuity from below or, m.c.f.b, in short) for any collection \(\{A_n\}_{n\geq 1}\)
of sets in \(\mc{F}\) such that \(A_n\subset A_{n+1}\) for all \(n\geq 1\) and \(\bigcup_{n\geq 1}A_n\in \mc{F}\),
\(\mu(\bigcup_{n\geq 1}A_n)=\lim_{n\to \infty}\mu(A_n)\).
\end{enumerate}

\vspace{2mm}
\textbf{Proof: } let \(\mu\) be a measure on \(\mc{F}\). Since \(\mu\) satisfies (iii), taking
\(A_3, A_4, \cdots, A_n = \emptyset\), \(\mu(A_1\cup A_2)=\mu(A_1)+\mu(A_2)\). Hence, if
\(A\subset B\) and \(A, B \in \mc{F}\), \(\mu(B)=\mu(A) + \mu(B\setminus A)\geq \mu(A)\), namely monotone. 
So, for \(A_1, A_2, \cdots, A_n \in \mc{F}\), if one \(\mu(A_n)=\infty\), \(\mu(\bigcup_{n\geq 1}A_n)=\lim_{n\to \infty}\mu(A_n)
=\infty\). Suppose all \(\mu(A_n) \neq \infty\), let \(B_n=(A_n\setminus A_{n-1})\) (take \(A_0=\emptyset\)), then
\(B_1, B_2, \cdots, B_n\) are disjoint and \(\bigcup_{n\geq 1}A_n=\bigcup_{n\geq 1}B_n\),
So \(\mu(\bigcup_{n\geq 1}A_n)=\mu(\bigcup_{n\geq 1}B_n)\), with (iii), \(\mu(\bigcup_{n\geq 1}B_n)
=\sum_{n=1}^{\infty}\mu(B_n)=\sum_{n=1}^{\infty}\mu(A_n-A_{n-1})=\lim_{n\to \infty}\mu(A_n)\). (ii)\('\) proved.

\vspace{1mm}
Coversely, let \(\{B_n\}\) a disjoint collection in \(\mc{F}\), take \(A_n=\bigcup_{n\geq 1} B_{n}\), then
\(A_n \subset A_{n+1}\) and
\(\bigcup_{n\geq 1}A_n=\bigcup_{n\geq 1}B_n\). With (ii)\('\), \(\mu(\bigcup_{n\geq 1}B_n)=\mu(\bigcup_{n\geq 1}A_n)
=\lim_{n\to\infty}\mu(A_n)\), with (i), \(\mu(A_n)=\sum_{i=1}^n\mu(B_i)\), then \(\mu(\bigcup_{n\geq 1}B_n)
=\lim_{n\to\infty}\sum_{i=1}^n\mu(B_n)=\sum_{n=1}^{\infty}\mu(B_n)\). Proved.

\vspace{2mm}
\textbf{Definition 1.2.2: } a measure \(\mu\) is called \textit{finite} or \textit{infinite} according as 
\(\mu(\Omega)<\infty\) or \(\mu(\Omega)=\infty\). A finite measure is called \textit{probability measure} if
\(\mu(\Omega)=1\). A measure \(\mu\) is called a \(\sifin\) if there exists a countable collection of sets
\(A_1, A_2, \cdots \in \mc{F}\), not necessarily disjoint, such that:

\begin{enumerate}[label=(\alph*)]
\item \(\bigcup_{n\geq 1}A_n=\Omega\).
\item \(\mu(A_n)<\infty\) for all \(n\geq 1\).
\end{enumerate}

\vspace{2mm}
\textbf{Proposition 1.2.2: } let \(\mu\) be a measure on an algebra \(\mc{F}\), and let \(A, B, A_1, A_2, \cdots, A_k\in \mc{F}\),
\(1\leq k < \infty\). Then:

\begin{enumerate}[label=(\roman*)]
\item (monotonicity) \(\mu(A)\leq\mu(B)\) if \(A\subset B\).
\item (finite subadditivity) \(\mu(A_1\cup A_2\cup \cdots\cup A_k )
\leq \sum_{i=1}^k\mu(A_i)\).
\item (inclusion-exclusion formula) if \(\mu(A_i)<\infty\) for all \(i=1,\cdots, k\), then
\(\mu(A_1\cup A_2\cup \cdots\cup A_k)=\sum_{i=1}^k\mu(A_k)-\sum_{1\leq i < j \leq k}
\mu(A_i\cap A_j)+\cdots+(-1)^{k-1}\mu(A_1\cap A_2\cap\cdots\cap A_k)\).
\end{enumerate}

\vspace{2mm}
\textbf{Proof: } if \(A\subset B\), \(A\cup (B\setminus) A = B\). With (iii) of \textbf{definition 1.2.1},
\(\mu(B)=\mu(A) + \mu(B\setminus A)\). As \(\mu(B\setminus A)\geq 0\),
\(\mu(B)\geq \mu(A)\). Meanwhile, suppose one of \(\mu(A_i), \mu(A_j) < \infty\),
then with (i), \(\mu(A_i\cap A_j)<\infty\) and \(\mu(A_i\cap A_j)-\mu(A_i\cap A_j)\)
is well defined. 
\(\mu(A_i\cup A_j)=\mu(A_i\cup (A_j\setminus A_i))=\mu(A_i)+\mu(A_j\setminus A_i)
=\mu(A_i)+(\mu(A_j\setminus A_i) + \mu(A_i\cap A_j)) - \mu(A_\cap A_j)=\mu(A_i)+\mu(A_j)-\mu(A_i\cap A_j)\).
i.e. \(\mu(A_i\cup A_j)\geq \mu(A_i)+\mu(A_j)\).
For (iii), it obviously holds when \(k=2\). Assuming (iii) holds for \(n\), then for \(n+1\):

\begin{equation*}
\begin{split}
&\mu(\bigcup_{i=1}^{n+1}A_i)=\mu(\bigcup_{i=1}^nA_i)+\mu(A_{n+1})
-\mu[(\bigcup_{i=1}^nA_i)\cap A_{n+1}]\\
&=\sum_{i=1}^n\mu(A_i)-\sum_{1\leq i <j\leq n}(A_i\cap A_j) + \cdots + 
(-1)^{n-1}\mu(A_1\cap A_2\cap\cdots\cap A_n)\\
&+\mu(A_{n+1})-\mu[\bigcup_{i=1}^n(A_i\cap A_{n+1})]\\
&=\sum_{i=1}^{n+1}\mu(A_i)-\sum_{1\leq i <j\leq n}(A_i\cap A_j) + \cdots + 
(-1)^{n-1}\mu(A_1\cap A_2\cap\cdots\cap A_n)\\
&-\sum_{i=1}^n\mu(A_i\cap A_{n+1})+\sum_{1\leq i < j \leq n}\mu(A_i\cap A_j\cap A_{n+1})
-\cdots\\&-(-1)^{n-1}\mu(A_1\cap A_2\cap\cdots\cap A_{n+1})\\
&=\sum_{i=1}^{n+1}\mu(A_i)-\sum_{i\leq i < j\leq n+1}(A_i\cap A_j)+\cdots+
(-1)^k\mu(A_1\cap A_2\cap\cdots\cap A_{n+1})
\end{split}
\end{equation*}

Proved.

\end{document}
